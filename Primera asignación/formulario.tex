\documentclass[11pt, letterpaper]{book}
\usepackage[utf8]{inputenc}
\usepackage[spanish]{babel}
\usepackage{amsmath, amssymb, amsfonts}

\title{Formulario Arquitectura del computador}
\author{Guillermo Cedeño}
\date{\today}

\begin{document}
\section{Formulario Arquitectura}\\

\subsection{Prestaciones}

$\displaystyle Prestaciones_x = \frac{1}{\text{Tiempo de ejecución}_x}$\\\\

"X es n veces más rápida que Y", si\\


$\displaystyle n = \frac{\text{Prestaciones}_X}{\text{Prestaciones}_Y}$

\subsection{Prestaciones De la CPU}\\

$\displaystyle \text{Tiempo de ejecución de CPU} = \text{Ciclos de reloj de la CPU }\times{\text{ Tiempo del
ciclo del reloj}}$\\\\

Tomando en cuenta \\

$\displaystyle\frac{1}{\text{Frecuencia del reloj}} = \text{Tiempo del ciclo del reloj}$\\\\

$\displaystyle \text{Tiempo de ejecución de CPU} = \frac{\text{Ciclos de reloj de la CPU }}{\text{Frecuencia de Reloj}}$\\\\


$(CPI)\text{Ciclos de reloj de CPU} = \text{Instrucciones de un programa } \times {\text{ Media de ciclos por instrucción}}$\\\\
Con lo cual tenemos\\

$\text{Tiempo de ejecución} = \text{Numero de instrucciones } \times {\text{ CPI }}\times{\text{ Tiempo de ciclo}}$\\

o bien, dado que la frecuencia es el inverso del tiempo del ciclo\\

$\displaystyle\text{Tiempo de ejecución} = \frac{\text{Numero de instrucciones } \times {\text{ CPI }}}{\text{Frecuencia de reloj}}$\\

También\\

$\displaystyle CPI = \frac{\text{Ciclos de reloj de la CPU}}{\text{Número de instrucciones}}$\\\\

$\displaystyle Tiempo = \frac{Segundos}{Programa} = \frac{Instrucciones}{Programa}\times\frac{\text{Ciclos de reloj}}{\text{Instrucción}}\times\frac{Segundos}{\text{Ciclo de reloj}}  $

\subsection{Circuitos integrados}\\

$\displaystyle\text{Coste por dado} = \frac{\text{Coste por oblea}}{\text{dado por oblea}\times{\text{Factor de producción}}}$\\\\

$\displaystyle\text{Dados por oblea} = \frac{\text{área de la oblea}}{\text{área del dado}}$\\\\

$\displaystyle\text{Factor de producción} = \frac{1}{( 1 +\text{Defectos por área}\times{\frac{\text{área del dado}}{2}})^2}$\\\\

\subsection{Razones SPEC}

Media geométrica\\\\

$\displaystyle\sqrt[n]{\prod_{i=1}^{n}\text{Relaciones de tiempo de ejecución}_i}$\\

$\displaystyle\text{ssj\_ops global por vatio} = \frac{(\sum_{i=0}^{10}ssj\_ops_i)}{(\sum_{i=0}^{10}potencia_i)}$\\

\subsection{Mejoras}

$\displaystyle\text{Tiempo de ejecución después de las mejoras} = \frac{\text{tiempo de ejecución por la mejora}}{\text{Cantidad de mejora}}+ \text{Tiempo de ejecución no afectado}$\\

\subsection{MIPS}

$\displaystyle MIPS=\frac{\text{Número de instrucciones}}{\text{tiempo de ejecución}\times 10^6}$\\\\

$\displaystyle MIPS=\frac{\text{número de instrucciones}}{\frac{\text{número de instrucciones}\times CPI}{\text{frecuencia de reloj}}\times 10^6} = \frac{\text{frecuencia de reloj}}{CPI\times 10^6}$

\end{document}